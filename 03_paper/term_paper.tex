\documentclass[11pt,]{article}
\usepackage{lmodern}

\usepackage{amssymb,amsmath}
\usepackage{ifxetex,ifluatex}
\usepackage{fixltx2e} % provides \textsubscript
\ifnum 0\ifxetex 1\fi\ifluatex 1\fi=0 % if pdftex
  \usepackage[T1]{fontenc}
  \usepackage[utf8]{inputenc}
\else % if luatex or xelatex
  \ifxetex
    \usepackage{mathspec}
    \usepackage{xltxtra,xunicode}
  \else
    \usepackage{fontspec}
  \fi
  \defaultfontfeatures{Mapping=tex-text,Scale=MatchLowercase}
  \newcommand{\euro}{€}
\fi
% use upquote if available, for straight quotes in verbatim environments
\IfFileExists{upquote.sty}{\usepackage{upquote}}{}
% use microtype if available
\IfFileExists{microtype.sty}{%
\usepackage{microtype}
\UseMicrotypeSet[protrusion]{basicmath} % disable protrusion for tt fonts
}{}
\usepackage[lmargin=5cm,rmargin=2.5cm,tmargin=2.5cm,bmargin=2.5cm]{geometry}

%% citation setup

\usepackage{csquotes}

\usepackage[backend=biber, maxbibnames = 99, style = apa]{biblatex}
\setlength\bibitemsep{1.5\itemsep}
\bibliography{references.bib}
\usepackage{color}
\usepackage{fancyvrb}
\newcommand{\VerbBar}{|}
\newcommand{\VERB}{\Verb[commandchars=\\\{\}]}
\DefineVerbatimEnvironment{Highlighting}{Verbatim}{commandchars=\\\{\}}
% Add ',fontsize=\small' for more characters per line
\usepackage{framed}
\definecolor{shadecolor}{RGB}{248,248,248}
\newenvironment{Shaded}{\begin{snugshade}}{\end{snugshade}}
\newcommand{\KeywordTok}[1]{\textcolor[rgb]{0.13,0.29,0.53}{\textbf{#1}}}
\newcommand{\DataTypeTok}[1]{\textcolor[rgb]{0.13,0.29,0.53}{#1}}
\newcommand{\DecValTok}[1]{\textcolor[rgb]{0.00,0.00,0.81}{#1}}
\newcommand{\BaseNTok}[1]{\textcolor[rgb]{0.00,0.00,0.81}{#1}}
\newcommand{\FloatTok}[1]{\textcolor[rgb]{0.00,0.00,0.81}{#1}}
\newcommand{\ConstantTok}[1]{\textcolor[rgb]{0.00,0.00,0.00}{#1}}
\newcommand{\CharTok}[1]{\textcolor[rgb]{0.31,0.60,0.02}{#1}}
\newcommand{\SpecialCharTok}[1]{\textcolor[rgb]{0.00,0.00,0.00}{#1}}
\newcommand{\StringTok}[1]{\textcolor[rgb]{0.31,0.60,0.02}{#1}}
\newcommand{\VerbatimStringTok}[1]{\textcolor[rgb]{0.31,0.60,0.02}{#1}}
\newcommand{\SpecialStringTok}[1]{\textcolor[rgb]{0.31,0.60,0.02}{#1}}
\newcommand{\ImportTok}[1]{#1}
\newcommand{\CommentTok}[1]{\textcolor[rgb]{0.56,0.35,0.01}{\textit{#1}}}
\newcommand{\DocumentationTok}[1]{\textcolor[rgb]{0.56,0.35,0.01}{\textbf{\textit{#1}}}}
\newcommand{\AnnotationTok}[1]{\textcolor[rgb]{0.56,0.35,0.01}{\textbf{\textit{#1}}}}
\newcommand{\CommentVarTok}[1]{\textcolor[rgb]{0.56,0.35,0.01}{\textbf{\textit{#1}}}}
\newcommand{\OtherTok}[1]{\textcolor[rgb]{0.56,0.35,0.01}{#1}}
\newcommand{\FunctionTok}[1]{\textcolor[rgb]{0.00,0.00,0.00}{#1}}
\newcommand{\VariableTok}[1]{\textcolor[rgb]{0.00,0.00,0.00}{#1}}
\newcommand{\ControlFlowTok}[1]{\textcolor[rgb]{0.13,0.29,0.53}{\textbf{#1}}}
\newcommand{\OperatorTok}[1]{\textcolor[rgb]{0.81,0.36,0.00}{\textbf{#1}}}
\newcommand{\BuiltInTok}[1]{#1}
\newcommand{\ExtensionTok}[1]{#1}
\newcommand{\PreprocessorTok}[1]{\textcolor[rgb]{0.56,0.35,0.01}{\textit{#1}}}
\newcommand{\AttributeTok}[1]{\textcolor[rgb]{0.77,0.63,0.00}{#1}}
\newcommand{\RegionMarkerTok}[1]{#1}
\newcommand{\InformationTok}[1]{\textcolor[rgb]{0.56,0.35,0.01}{\textbf{\textit{#1}}}}
\newcommand{\WarningTok}[1]{\textcolor[rgb]{0.56,0.35,0.01}{\textbf{\textit{#1}}}}
\newcommand{\AlertTok}[1]{\textcolor[rgb]{0.94,0.16,0.16}{#1}}
\newcommand{\ErrorTok}[1]{\textcolor[rgb]{0.64,0.00,0.00}{\textbf{#1}}}
\newcommand{\NormalTok}[1]{#1}
\usepackage{longtable,booktabs}
\usepackage{graphicx}
\makeatletter
\def\maxwidth{\ifdim\Gin@nat@width>\linewidth\linewidth\else\Gin@nat@width\fi}
\def\maxheight{\ifdim\Gin@nat@height>\textheight\textheight\else\Gin@nat@height\fi}
\makeatother
% Scale images if necessary, so that they will not overflow the page
% margins by default, and it is still possible to overwrite the defaults
% using explicit options in \includegraphics[width, height, ...]{}
\setkeys{Gin}{width=\maxwidth,height=\maxheight,keepaspectratio}
\ifxetex
  \usepackage[setpagesize=false, % page size defined by xetex
              unicode=false, % unicode breaks when used with xetex
              xetex]{hyperref}
\else
  \usepackage[unicode=true]{hyperref}
\fi
\hypersetup{breaklinks=true,
            bookmarks=true,
            pdfauthor={Jonathan Berrisch, Timo Rammert},
            pdftitle={Term Paper: Statistical Learning},
            colorlinks=true,
            citecolor=blue,
            urlcolor=blue,
            linkcolor=magenta,
            pdfborder={0 0 0}}
\urlstyle{same}  % don't use monospace font for urls
\setlength{\parindent}{0pt}
\setlength{\parskip}{6pt plus 2pt minus 1pt}
\setlength{\emergencystretch}{3em}  % prevent overfull lines
\setcounter{secnumdepth}{5}

%%% Use protect on footnotes to avoid problems with footnotes in titles
\let\rmarkdownfootnote\footnote%
\def\footnote{\protect\rmarkdownfootnote}

%%% Change title format to be more compact
\usepackage{titling}

% Create subtitle command for use in maketitle
\newcommand{\subtitle}[1]{
  \posttitle{
    \begin{center}\large#1\end{center}
    }
}

\setlength{\droptitle}{-2em}
  \title{Term Paper: Statistical Learning}
  \pretitle{\vspace{\droptitle}\centering\huge}
  \posttitle{\par}
\subtitle{Subtitle}
  \author{Jonathan Berrisch, Timo Rammert}
  \preauthor{\centering\large\emph}
  \postauthor{\par}
  \predate{\centering\large\emph}
  \postdate{\par}
  \date{today}


%% linespread settings

\usepackage{setspace}

\onehalfspacing

% Language Setup

\usepackage{ifthen}
\usepackage{iflang}
\usepackage[super]{nth}
\usepackage[ngerman, english]{babel}

\begin{document}

\selectlanguage{english}


%\maketitle

\begin{titlepage}
  \noindent\begin{minipage}{0.6\textwidth}
	  \IfLanguageName{english}{University of Duisburg-Essen}{Universität Duisburg-Essen}\\
	  \IfLanguageName{english}{Faculty of Business Administration and Economics}{Fakultät für Wirtschaftswissensschaften}\\
	  \IfLanguageName{english}{Chair of Econometrics}{Lehrstuhl für Ökonometrie}\\
  \end{minipage}
	\begin{minipage}{0.4\textwidth}
	  \begin{flushright}
  	  \vspace{-0.5cm}
      \IfLanguageName{english}{\includegraphics*[width=5cm]{Includes/duelogo_en.png}}{\includegraphics*[width=5cm]{Includes/duelogo_de.png}}
	  \end{flushright}
	\end{minipage}
  \\
  \vspace{1.5cm}
  \begin{center}
  \huge{Term Paper: Statistical Learning}\\
  \vspace{.25cm}
  \Large{Subtitle}\\
  \vspace{0.5cm}
  \large{Type of Paper}\\
  \vspace{1cm}
  \large{
  \IfLanguageName{english}{Submitted to the Faculty of \\ Business Administration and Economics \\at the \\University of Duisburg-Essen}{Vorgelegt der \\Fakultät für Wirtschaftswissenschaften der \\ Universität Duisburg-Essen}\\}
  \vspace{0.75cm}
  \large{\IfLanguageName{english}{from:}{von:}}\\
  \vspace{0.5cm}
  Jonathan Berrisch, Timo Rammert\\
  \end{center}
  \vspace{4cm}

  \noindent\begin{minipage}[t]{0.3\textwidth}
  \IfLanguageName{english}{Matriculation Number}{Matrikelnummer}
  \end{minipage}
  \begin{minipage}[t]{0.7\textwidth}
  \hspace{1cm}Matriculation Number
  \end{minipage}

  \noindent\begin{minipage}[t]{0.3\textwidth}
  \IfLanguageName{english}{Study Path:}{Studienfach:}
  \end{minipage}
  \begin{minipage}[t]{0.7\textwidth}
  \hspace{1cm}Study Path
  \end{minipage}

  \noindent\begin{minipage}[t]{0.3\textwidth}
  \IfLanguageName{english}{Reviewer:}{Erstgutachter:}
  \end{minipage}
  \begin{minipage}[t]{0.7\textwidth}
  \hspace{1cm}Prof.~Dr.~Christoph Hanck
  \end{minipage}

  \noindent\begin{minipage}[t]{0.3\textwidth}
  \IfLanguageName{english}{Secondary Reviewer:}{Zweitgutachter}
  \end{minipage}
  \begin{minipage}[t]{0.7\textwidth}
  \hspace{1cm}Prof.~Dr.~Andreas Behr
  \end{minipage}

  \noindent\begin{minipage}[t]{0.3\textwidth}
  Semester:
  \end{minipage}
  \begin{minipage}[t]{0.7\textwidth}
  \hspace{1cm}\IfLanguageName{english}{\nth{1} Semester}{1. Fachsemester}
  \end{minipage}

  \noindent\begin{minipage}[t]{0.3\textwidth}
  \IfLanguageName{english}{Graduation (est.):}{Vsl. Studienabschluss:}
  \end{minipage}
  \begin{minipage}[t]{0.7\textwidth}
  \hspace{1cm}Summer Term 2019
  \end{minipage}

  \noindent\begin{minipage}[t]{0.3\textwidth}
  \IfLanguageName{english}{Deadline:}{Abgabefrist:}
  \end{minipage}
  \begin{minipage}[t]{0.7\textwidth}
  \hspace{1cm}Deadline
  \end{minipage}

\end{titlepage}


\pagenumbering{Roman} 
{
\hypersetup{linkcolor=black}
\setcounter{tocdepth}{3}
\tableofcontents
}
\newpage
\listoftables
\newpage
\listoffigures
\newpage
\pagenumbering{arabic} 
\section{Introduction}\label{introduction}

\subsection{Data}\label{data}

The data used in this term paper is taken from a wine themed dataset.
The variables include among others the name of the wine, the country of
origin, the price, the amount (litres), the taste segment and some
variables about different reviews.

\section{Analysis}\label{analysis}

At first a mean regression and a linear regression are calculated and
used as a baseline for further methods. For comparison of the different
models, the Root Mean Sqaured Error (RMSE)
\[\sqrt{\frac{1}{n}\sum_{i = 1}^{n}\left(y_i-\hat{y}_i\right)^2}\] is
calculated.

\includegraphics{term_paper_files/figure-latex/Mean Regression-1.pdf}

\begin{verbatim}
##        mod    rmse
## 1     Mean 2193578
## 2 Basic_lm      NA
\end{verbatim}

\subsection{R Markdown}\label{r-markdown}

This is an R Markdown document. Markdown is a simple formatting syntax
for authoring HTML, PDF, and MS Word documents. For more details on
using R Markdown see \url{http://rmarkdown.rstudio.com}.

When you click the \textbf{Knit} button a document will be generated
that includes both content as well as the output of any embedded R code
chunks within the document. You can embed an R code chunk like this:

\begin{Shaded}
\begin{Highlighting}[]
\KeywordTok{summary}\NormalTok{(cars)}
\end{Highlighting}
\end{Shaded}

\begin{verbatim}
##      speed           dist       
##  Min.   : 4.0   Min.   :  2.00  
##  1st Qu.:12.0   1st Qu.: 26.00  
##  Median :15.0   Median : 36.00  
##  Mean   :15.4   Mean   : 42.98  
##  3rd Qu.:19.0   3rd Qu.: 56.00  
##  Max.   :25.0   Max.   :120.00
\end{verbatim}

\subsection{Including Plots}\label{including-plots}

You can also embed plots, for example:

\begin{figure}
\centering
\includegraphics{term_paper_files/figure-latex/pressure-1.pdf}
\caption{\label{fig:pressure} Pressure}
\end{figure}

Note that the \texttt{echo\ =\ FALSE} parameter was added to the code
chunk to prevent printing of the R code that generated the plot. You can
label the plot above by including the label
\texttt{\textbackslash{}\textbackslash{}label\{fig:pressure\}} in the
chunk argument \texttt{fig.cap}. A reference to the plot is then made as
follows:

Looking at Figure \ref{fig:pressure} makes me happy.

\subsection{The YAML Header}\label{the-yaml-header}

The YAML Header is at the very top of this document. It is enclosed in 3
dashes. Although there are already some options specified you might want
to change some additional things. Here are some useful things that we
have implemented for you.

\subsubsection{language}\label{language}

You can either set this to english or german. If the language is not
specified in the YAML header it will use english e.g.:

language: english

language: german

This variable affects mainly the headings of your output file.

\subsubsection{linespread}\label{linespread}

The default value for the linespread is 1.5. Usually this is fine and
sometimes it's required. If you nevertheless want to change it you can
do so by specifying the linespread variable. E.g.:

linespread: 1

\section{How Rmarkdown makes your life
easy}\label{how-rmarkdown-makes-your-life-easy}

\subsection{The kable function}\label{the-kable-function}

In Empirical work it's crucial not only to present your results but also
to explain your research strategy. Often times that involes creating
tables to present the used data and creating tables to show your
results. Creating tables (e.g.~in Latex) by hand is hard and time
consuming but there are a variety of packages out there that can
automate this job for you. One of them is the kable package. It can
produce Latex tables from a variety of R Objects.

Consider the following Example: you are working on an analysis of black
cherry trees and want to present n observations to the reader You can do
that using the knitr::kable() function.

\begin{longtable}[]{@{}rrr@{}}
\caption{6 Observations from the trees Dataset}\tabularnewline
\toprule
Diameter & Height & Volume\tabularnewline
\midrule
\endfirsthead
\toprule
Diameter & Height & Volume\tabularnewline
\midrule
\endhead
12.0 & 75 & 19.1\tabularnewline
11.0 & 75 & 18.2\tabularnewline
8.3 & 70 & 10.3\tabularnewline
17.3 & 81 & 55.4\tabularnewline
13.8 & 64 & 24.9\tabularnewline
11.7 & 69 & 21.3\tabularnewline
\bottomrule
\end{longtable}

Now that we have presented our data it's analysis time! Lets start with
a quick call to summary().

\subsection{The Stargazer package}\label{the-stargazer-package}

Calling summary() in a code chunk will work but this will give you quite
an ugly result (just try it for yourself!). When it comes to presenting
more structured objects like summaries, model results or for example
correlation matrices the stargazer package is well suited.

\begin{table}[!htbp] \centering 
  \caption{Summary} 
  \label{} 
\begin{tabular}{@{\extracolsep{5pt}}lccccccc} 
\\[-1.8ex]\hline 
\hline \\[-1.8ex] 
Statistic & \multicolumn{1}{c}{N} & \multicolumn{1}{c}{Mean} & \multicolumn{1}{c}{St. Dev.} & \multicolumn{1}{c}{Min} & \multicolumn{1}{c}{Pctl(25)} & \multicolumn{1}{c}{Pctl(75)} & \multicolumn{1}{c}{Max} \\ 
\hline \\[-1.8ex] 
Girth & 31 & 13.248 & 3.138 & 8.300 & 11.050 & 15.250 & 20.600 \\ 
Height & 31 & 76.000 & 6.372 & 63 & 72 & 80 & 87 \\ 
Volume & 31 & 30.171 & 16.438 & 10.200 & 19.400 & 37.300 & 77.000 \\ 
\hline \\[-1.8ex] 
\end{tabular} 
\end{table}

Assume we want to evaluate how the height and the volume of a typical
cherry tree are related. We are estimating this using OLS to estimate a
simple linear model.

\begin{table}[!htbp] \centering 
  \caption{Regression results} 
  \label{} 
\begin{tabular}{@{\extracolsep{5pt}}lc} 
\\[-1.8ex]\hline 
\hline \\[-1.8ex] 
 & \multicolumn{1}{c}{\textit{Dependent variable:}} \\ 
\cline{2-2} 
\\[-1.8ex] & Volume \\ 
\hline \\[-1.8ex] 
 Height & 1.543$^{***}$ \\ 
  & (0.384) \\ 
  & \\ 
 Constant & $-$87.124$^{***}$ \\ 
  & (29.273) \\ 
  & \\ 
\hline \\[-1.8ex] 
Observations & 31 \\ 
R$^{2}$ & 0.358 \\ 
Adjusted R$^{2}$ & 0.336 \\ 
Residual Std. Error & 13.397 (df = 29) \\ 
F Statistic & 16.164$^{***}$ (df = 1; 29) \\ 
\hline 
\hline \\[-1.8ex] 
\textit{Note:}  & \multicolumn{1}{r}{$^{*}$p$<$0.1; $^{**}$p$<$0.05; $^{***}$p$<$0.01} \\ 
\end{tabular} 
\end{table}

Now that we have our model we can visualize it.

\begin{figure}
\centering
\includegraphics{term_paper_files/figure-latex/unnamed-chunk-4-1.pdf}
\caption{Dataset and regression}
\end{figure}

\pagebreak

\subsection{Citations}\label{citations}

The a \emph{bibtex} bibliography can be used for citations. The file
name of the bibliography used in this template is
\texttt{references.bib}

You can cite a source like this: \autocite{Keil_2012} or
\textcite{Litterman1986}.

A cited source will be automatically added to the end of the document.

\pagebreak
\renewcommand*{\mkbibnamefamily}[1]{\textbf{#1}}
\renewcommand*{\mkbibnamegiven}[1]{\textbf{#1}}
\renewcommand*{\mkbibnameprefix}[1]{\textbf{#1}}
\renewcommand*{\mkbibnamesuffix}[1]{\textbf{#1}}
\printbibliography[title=References]

\newpage
\textbf{Eidesstattliche Versicherung}

\bigskip

Ich versichere an Eides statt durch meine Unterschrift, dass ich die vorstehende Arbeit selbständig und ohne fremde Hilfe angefertigt und alle Stellen, die ich wörtlich oder annähernd wörtlich aus Veröffentlichungen entnommen habe, als solche kenntlich gemacht habe, mich auch keiner anderen als der angegebenen Literatur oder sonstiger Hilfsmittel bedient habe. Die Arbeit hat in dieser oder ähnlicher Form noch keiner anderen Prüfungsbehörde vorgelegen.

\vspace{1cm}
\rule{0pt}{2\baselineskip} %
\par\noindent\makebox[2.25in]{\indent Essen, den \hrulefill} \hfill\makebox[2.25in]{\hrulefill}%
\par\noindent\makebox[2.25in][l]{} \hfill\makebox[2.25in][c]{Jonathan Berrisch, Timo Rammert}%


\end{document}
